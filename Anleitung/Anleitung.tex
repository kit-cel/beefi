\documentclass[a4paper,twoside]{scrreprt}
\usepackage[utf8]{inputenc}
\usepackage[ngerman]{babel}
\usepackage[]{tikz}
\usepackage[]{pgfplots}
\usepgfplotslibrary{polar}
\usepackage[locale=DE]{siunitx}
\DeclareSIUnit{\dbm}{dBm}
\usepackage{hyperref}
\usepackage{microtype}
\usepackage{glossaries}
\usepackage{tabularx}
\usepackage{graphicx}
\title{Benutzung des kontrollierbaren WLAN-Bestrahlungssystems \textsc{Beefi}}
\subject{Nachrichtentechnischer Bericht}
\author{M.Sc. Marcus Müller}
\publishers{Institut für Nachrichtentechnik (CEL)}
\titlehead{\raggedleft\includegraphics[height=4em]{cel.pdf}} 
\newacronym{cel}{CEL}{Institut für Nachrichtentechnik}
\newacronym{sdr}{SDR}{Software Defined Radio}
\newacronym{eirp}{EIRP}{Equivalent Isotropic Radiated Power, Äquivalente
  isotrope abgestrahlte Leistung}
\newacronym{wlan}{WLAN}{Wireless LAN}
\makeglossaries
\newcommand{\lowf}{\SI{2.450}{\giga\hertz}}
\newcommand{\highf}{\SI{5.805}{\giga\hertz}}
\begin{document}
\maketitle
\chapter{Systembeschreibung}

Beim Beefi-Aufbau handelt es sich um ein System zur Bereitstellung einer
kontrollierten Strahlungsleistung und damit auch Bestrahlungsintensität. Hierbei
werden Wellenformen eingesetzt, die \gls{wlan}-Übertragungen entsprechen.

Technisch herrscht kein Unterschied zwischen den von einem \gls{wlan}-Gerät
ausgesandten Kommunikationssignalen und den synthetisierten Wellenformen: Es
handelt sich um gültige \gls{wlan}-Pakete nach dem IEEE802.11a-Standard. Der
technische Unterschied besteht darin, dass der Sender nicht versucht, eine
Kommunikationsstrecke zu betreiben, und hierbei Leistung, Paketdichte und
eventuell Strahlrichtung während des Betriebs dynamisch anzupassen.

Stattdessen setzt das Beefi-System \gls{sdr}-Technologie ein um eine
einstellbare, zuverlässige, quantisierbare und protokollierte Bestrahlung
sicherzustellen, wie sie im Sinne reproduzierbarer Wissenschaft notwendig ist.

Hierfür sind drei Parameter einstellbar:

\begin{enumerate}
\item Die Sendeverstärkung über einen Bereich von $1:10^9$
\item Die Auslastung mit WLAN-Paketen
\item Die Dauer der Periode, in die die Zeit eingeteilt wird.
\end{enumerate}

Der zentrale Laptop im System dient dabei der Ansteuerung der Sendegeräte und
der Visualisierung der aktuellen Sendeeinstellung. Außerdem protokolliert er,
wann mit welcher Parametrisierung bestrahlt wurde, und bietet mit seiner
Batterie eine Robustheit gegen kurzfristige Unterbrechungen der
Stromversorgung.

\section{Richtdiagramme}

\subsection{Mathematisch-Physikalische Grundlagen}
\autoref{fig:richt} stellt die Richtdiagramme des Aufbaus bei \lowf{} und \highf{}
dar. Hierbei wird die Flächenleistungsdichte in Bezug auf die Hauptrichtung in
Dezibel angeben. Bei Leistungsgrößen ist das Dezibel definiert als

\begin{align}
x_{\text{dB}}&=10\log_{10}\frac{x_{\text{[linear]}}}{x_{\text{Referenz}}}\text.
\end{align}

Ein Eintrag von \SI{-30}{\decibel} bedeutet also, dass bezogen auf die
Hauptstrahlrichtung (\SI{0}{\degree}) nur $10^{-3}$, ein Tausendstel, an
Leistung auf eine gleich große Fläche fällt.

Im Folgenden wird die Einheit \si{\decibel\milli{}} verwandt, das ist
\emph{Dezibel bezogen auf ein Milliwatt}; \SI{20}{\decibel\milli{}} entsprechen
also $\SI{1e2}{\milli\watt}=\SI{100}{\milli\watt}$.

Die Werte wurden mittels eines kalibrierten Spektrumsanalysator (Rohde\&Schwarz
FSQ 8) ermittelt. Hierbei wurde eine entsprechende richtungsunabhängige
Dual-Band-Antenne eingesetzt.

Da die im Richtdiagramm angegebenen Werte gemessen wurden (die
Messstandardabweichung lag bei Betrachtung in Dezibel bei 0.45), wurde zur
Reduktion der Messungenauigkeit gegenüberliegende Messungen gemittelt und für
beide Seiten verwendet, wodurch eine Punktsymmetrie dieses Diagramms zustande
kommt. Vorher fand eine Verifikation statt, dass die seitenabhängigen
Unterschiede unterhalb der Messgenauigkeit liegen.

Durch direkte Messung der Ausgangsleistung der Sender wurde die insgesamt
ausgestrahlte Leistung bestimmt. Da die verwendeten Antennen eine
vernachlässigbare Abweichung von perfekter Abstrahleffizienz haben, lässt sich
zusammen aus dem Richtdiagramm das EIRP bestimmen.

\begin{figure}[pbt]
  \centering
  \begin{tikzpicture}[]
    \begin{polaraxis}[scale=2.25,
      xticklabel=$\pgfmathprintnumber{\tick}^\circ$,
      yticklabel=$\pgfmathprintnumber{\tick}\,\text{dB}$,
      xtick={0,30,...,330},
      ytick={-30,-20,...,0},
      ymin=-31, ymax=4,
      y coord trafo/.code=\pgfmathparse{65+#1},
      rotate=-90,
      y coord inv trafo/.code=\pgfmathparse{#1-65},
      x dir=reverse,
      xticklabel style={anchor=-\tick-90},
      yticklabel style={anchor=east},
      % y axis line style={yshift=-4.75cm},
      % ytick style={yshift=-4.75cm}
      ]
      \addplot[color=red] table {
        0 0.00
        10 -3.50
        20 -3.50
        30 -9.50
        40 -8.50
        50 -16.50
        60 -15.50
        70 -14.50
        80 -18.50
        90 -29.50
        100 -29.50
        110 -17.50
        120 -13.50
        130 -10.50
        140 -6.50
        150 -4.50
        160 -1.50
        170 -1.50
        180 -0.50
        190 -3.50
        200 -3.50
        210 -9.50
        220 -8.50
        230 -16.50
        240 -15.50
        250 -14.50
        260 -18.50
        270 -29.50
        280 -29.50
        290 -17.50
        300 -13.50
        310 -10.50
        320 -6.50
        330 -4.50
        340 -1.50
        350 -1.50
        360 0
      };
      \addplot[color=blue] table {
        0 0
        10 -1.00
        20 -3.00
        30 -9.00
        40 -13.00
        50 -17.00
        60 -14.00
        70 -12.00
        80 -12.00
        90 -14.00
        100 -21.00
        110 -13.00
        120 -16.00
        130 -19.00
        140 -13.00
        150 -9.00
        160 -5.00
        170 -1.00
        180 0.00
        190 -1.00
        200 -3.00
        210 -9.00
        220 -13.00
        230 -17.00
        240 -14.00
        250 -12.00
        260 -12.00
        270 -14.00
        280 -21.00
        290 -13.00
        300 -16.00
        310 -19.00
        320 -13.00
        330 -9.00
        340 -5.00
        350 -1.00
        360 0.00
      };
      \addlegendentry{\lowf}
      \addlegendentry{\highf}
    \end{polaraxis}
  \end{tikzpicture}
  \caption{Richtdiagramme bei \lowf{} und \highf. Die Winkelangaben beziehen
    sich auf die am Gerät angebrachte Winkelmarkierungen.}
  \label{fig:richt}
\end{figure}

Bei Vollaussteuerung in Hauptstrahlrichtung eines
WLAN-Geräts mit einer äquivalenten isotropen Strahlungsleistung (\gls{eirp}) von
\SI{22.5}{\decibel\milli{}} bei \lowf{} und \SI{10.0}{\decibel\milli{}} bei
\highf pro Seite.

Damit liegen wir pro Seite bei \SI{2.25}{\decibel} oberhalb der nominell
zulässigen Abstrahlleistung von \SI{20}{\decibel\milli{}} im
\SI{2.4}{\giga\hertz}-ISM-Band. Im \SI{5}{\giga\hertz}-Band sind die maximal
zulässigen Leistungen wesentlich höher
($\SI{1}{\watt}=\SI{30}{\decibel\milli{}}$), das ist allerdings der höheren
Freiraumdämpfung geschuldet.

Die empfangene Leistung bei einer Frequenz nimmt mit den Quadraten des Abstands
und der Frequenz ab. Die Freiraumdämpfung  $a$ bei der Frequenz $f$ im Abstand $r$ ist 

\begin{align}
  a_{\text{Freiraum}} &= \left( \frac{4\pi}{c}\right)^2\cdot r^2f^2\text{,} \label{eq:fspl}
      \intertext{
      Für die beiden betrachteten Frequenzen beträgt diese Dämpfung in Dezibel:
      }
  a_{\lowf} &= \left( \num{40.23} + 20\log_{10} r_{[\text{Meter}]}\right)\si{\decibel{}}\label{eq:fspl24}\\
  a_{\highf} &= \left( \num{47.72} + 20\log_{10} r_{[\text{Meter}]}\right)\si{\decibel{}}\label{eq:fspl58}\\
\end{align}

wobei $c\approx\SI{3e8}{\meter\per\second}$ die Licht- bzw.
Wellengeschwindigkeit ist, und gibt das Verhältnis empfangener Leistung zur gesendeten an.

\section{Leistungsdichten in festen Abstand}

\begin{figure}[bt]
  \centering

  \begin{tikzpicture}[x=3cm,y=-3cm]
    \node (point) at (0,0) {p};
    \node (rx) at (1,0) {RX};
    \node[] (tx) at (0,2) {TX};
    
    \node[anchor=north] at (0,2.1) {Zentrum der Sendeantenne};
    \node[anchor=east] at (-0.1,0) {\parbox{10em}{Lotpunkt auf Gerade rechtwinklig zur Geräteachse}\;};

    \draw[thick] (tx) -- node[right, midway]{$r$} (rx);
    \draw[->] (point) -- node[above] {$\Delta x$} (rx);
    \draw[] (tx) -- node[left] {$\Delta y=\SI{2}{\meter}$} (point);
  \end{tikzpicture}
  \caption{Geometrie des Aufbaus bei Platzierung der Bienenstöcke auf einer zur
    Geräteachse rechtwinkligen Linie im Abstand von \SI{2}{\meter}\label{fig:geo}}
\end{figure}

Aus der Geometrie des Problems (vergleiche \autoref{fig:geo}) ergibt sich ein
quadratischer Abstand
\begin{align}
  r^2=(\Delta y)^2+(\Delta x)^2\text.\label{eq:r}
\end{align}

Zum Bestimmen der relativen Leistungsdichte an einem beliebigen Punkt relativ
zum Mittelpunkt der Antenne muss nun folgendes Verfahren eingesetzt werden:
\begin{enumerate}
\item Messung oder Berechnung des Abstands $r$ gemäß \autoref{eq:r},
\item Bestimmen der Freiraumdämpfung in Dezibel mittels Einsetzen von $r$ in
  \autoref{eq:fspl}
\item Bestimmen der relativen Richtungsdämpfung in Dezibel aus dem Richtdiagramm
  \autoref{fig:richt} (Gegebenenfalls Berechnung des Winkels mittels
  $\alpha=\arctan\frac{\Delta x}{\Delta y}$)
\item Addition dieser Dämpfungen in Dezibel
\end{enumerate}

\paragraph{Beispiel} Bei \lowf{} ergibt sich im Abstand von \SI{2}{\meter}
direkt vor der Antenne (Punkt \glqq p\grqq{} in \autoref{fig:geo}) eine Dämpfung
von $(\num{40.23}+3)\,\si{\decibel}$ (da $\log_{10}(2) \approx 0.3$).\\
Bewegt man sich zwei Meter nach rechts (Punkt \grqq RX\glqq{} bei $\Delta
x=\SI{2}{\meter}$), wird $\alpha=\SI{45}{\degree}$.\\
Diese Richtung ist um etwa \SI{-12}{\decibel} gedämpft.\\
Gleichzeitig ist der Abstand jetzt $r=\sqrt8\,\si{m}$, und
$20\log_{10}\left(\sqrt8\right)\approx 9$, so dass sich insgesamt eine Dämpfung
von $(\num{40.23}+9)\si{\decibel}$ ergibt.\\
Relativ zum Punkt \glqq p\grqq{} sind das \SI{18}{\decibel} mehr, was einem
Faktor von
$10^{\frac{18}{10}}\approx 63$ weniger entspricht.

\paragraph{Beispiel 2} Bei \highf{} im Abstand von \SI{2}{\meter} direkt vor der
Antenne herrscht eine Dämpfung von $(\num{47.72}+3)\,\si{\decibel}$. Geht man vier
Meter linkswärts ($\Delta x=-\SI{4}{\meter}$), wird
$\alpha=\SI{-63}{\degree}\equiv\SI{297}{\degree}$, was laut Richtdiagramm um ca.
\SI{-19}{\decibel} gedämpft ist. Der Abstand beträgt nun $r=\sqrt{20}\,\si{m}$, und
$20\log_{10}\left(\sqrt{20}\right)\approx 13$, so dass sich insgesamt eine
Dämpfung relativ zum Punkt \glqq p\grqq{} von \SI{22}{\decibel} ergibt, was
einem Faktor von $159$ weniger entspricht.

Zu beachten ist, dass durch den logarithmischen Eingang des Abstands relativ
kleine Änderungen am $\Delta x$ kleine Auswirkungen haben; einen Bienenstock 10
oder 30\,\si{\centi\meter} von der Achse zu verrücken, um zwei Bienenstöcke in
die Hauptkeule zu bringen, ist also unkritisch.

\chapter{Bedienungsanleitung}
\section{Aufbau \& Inbetriebnahme}

Nach Verbindung der Stromversorgung und Einstecken der USB-Kabel muss der Laptop
nur aufgeklappt und eingeschaltet werden. Der Startvorgang ist vollautomatisch.

Der Rechner fährt hierbei hoch, lädt die nötige Firmware auf die beiden USRPs,
die sich unterhalb des Laptops befinden, und startet das Sendeprogramm. Dies
kann bis zu \SI{5}{\minute} dauern!

Sobald dieses läuft wird fortwährend protokolliert, mit welcher Ausgangsdämpfung
gesendet wurde.
\section{Start der Bestrahlung}

Zunächst ist die minimal Sendeverstärkung von \SI{0}{\decibel} eingestellt, was
relativ zur maximalen Verstärkung von \SI{90}{\decibel} einer Dämpfung von einer
Milliarde entspricht -- effektiv ist die Strahlungsquelle also noch nicht aktiv.

Im Drop-Down-Menu \texttt{Output Power} kann eine höhere Verstärkung gewählt
werden. Im Normalbetrieb wird die volle Sendeverstärkung von \SI{90}{\decibel}
am meisten Sinn ergeben. Mit dieser wurden die im vorhergehenden Kapitel
angegebenen Leistungen gemessen.

Der \texttt{Duty Cycle} gibt an, zu wie vielen Prozenten der Zeit WLAN-Pakete
gesendet werden. Die Zeit wird hierfür in Perioden der Länge
\texttt{Period}\,\si{\milli\second} eingeteilt, von denen die ersten
\texttt{Duty Cycle} \% mit WLAN-Paketen (die sehr viel kürzer sind) belegt wird,
und die letzten (100 - \texttt{Duty Cycle}) \% nicht gesendet wird.

So lassen sich unterschiedlich ausgelastete WLAN-Netze simulieren.

\section{Abruf des Bestrahlungsprotokolls}

Auf dem Laptop wird eine Textdatei mit kommaseparierten Werten (CSV) pro
Sendeband (also separat für \lowf{} und \highf) angelegt. Diese Dateien finden
sich im Ordner \texttt{/car/log/beefi} und tragen den Namen
\texttt{beefi\_log\_}Zeitstempel in Sekunden\texttt{.csv}.

Am elegantesten lässt sich per Netzwerk auf diese Dateien zugreifen.

Verbinden Sie dazu den Laptop mit einem Netzwerk, in dem ein DHCP-Server läuft
(also zum Beispiel ein Heimrouter, der IP-Adressen vergibt). Nach kurzer Zeit
sollte in der Fußzeile der Anzeige eine grüne Zahlenreihe (IP-Adresse), nach dem
Muster {\color{green}{\texttt{000.000.000.000}}} (IPv4) und/oder
{\color{green}{\texttt{2a00:1398:4:5e01:529d:4b8:5928:86fc}}}. Notieren Sie sich
diese Nummer (oder zumindest eine davon). Diese Adresse kann sich bei jedem
Neuverbinden mit dem Netzwerk ändern.

Sie können als Nutzer \texttt{beefi} mit dem Passwort \texttt{VeT8k12} auf diesen
Rechner zugreifen per SSH. Über das SCP-Protokoll können Dateien ausgetauscht
werden. Unter Windows und Mac OS X erfüllen verschiedene Programme die Funktion
eines SCP-Clients. \href{https://filezilla-project.org}{FileZilla} ist im
deutschen Sprachraum sehr beliebt. Sie können sich mit dem Laptop verbinden,
indem Sie auf Ihrem Rechner im selben Netzwerk Filezilla starten, bei
\texttt{Host:} die oben notierte IP-Adresse eingeben. Navigieren Sie auf der
\texttt{Remote Site} nach \texttt{/var/log/beefi/} und ziehen Sie die
Logdateien, die relevant sind, auf Ihren Rechner.

Die Dateien sind in Pandas, LibreOffice Calc, Google Sheets oder Microsoft Excel
importierbar. Wählen Sie hierbei als Feldtrenner \texttt{;} und, falls gefragt,
als Zeilenumbruch \emph{Unix}.

\section{Fernwartung}

Sobald der Laptop mit einem Netzwerk verbunden wird (zum Beispiel wie oben, oder
durch Verbindung mit einem Handy im USB-Tethering-Modus), versucht er, eine
Verbindung mit einem VPN aufzubauen. Bisher sind außer dem Laptop keinerlei
weitere Teilnehmer in diesem VPN. Sollte allerdings eine Fernwartung nötig
werden, kann ich mich mit diesem VPN verbinden, um dann per SSH auf den Rechner
zuzugreifen.

\end{document}